\documentclass[a4paper,12pt]{article} 
\usepackage{mathrsfs}
\usepackage[utf8]{inputenc}
\usepackage[spanish]{babel}
\usepackage{amsmath}
\usepackage{amsfonts}
\usepackage{amssymb} 
\usepackage{graphicx} 
\usepackage{hyperref} 
\usepackage{wrapfig}
\usepackage{enumitem}
\usepackage{fancyhdr}
\usepackage{float}
\usepackage{eurosym}
\usepackage{color}
\usepackage{circuitikz}
\usepackage{titling}
\usepackage{hyperref}
\usepackage{media9}
\usepackage{lipsum}
\usepackage{tocbibind}
\usepackage{listings}
\usepackage{tabularx}
\usepackage{tcolorbox}
\usepackage{bookmark}
\usepackage{media9}
\usepackage[table]{xcolor}
\definecolor{lightblue}{RGB}{228, 244, 253}
\usepackage{listings}
\usepackage{color}

\definecolor{dkgreen}{rgb}{0,0.6,0}
\definecolor{gray}{rgb}{0.5,0.5,0.5}
\definecolor{mauve}{rgb}{0.58,0,0.82}

\lstset{frame=tb,
  language=Python,
  inputencoding=utf8,
  extendedchars=true,
  aboveskip=3mm,
  belowskip=3mm,
  showstringspaces=false,
  columns=flexible,
  basicstyle={\small\ttfamily},
  numbers=none,
  numberstyle=\tiny\color{gray},
  keywordstyle=\color{blue},
  commentstyle=\color{dkgreen},
  stringstyle=\color{mauve},
  breaklines=true,
  breakatwhitespace=true,
  tabsize=3,
  literate=%
    {á}{{\'a}}1
    {é}{{\'e}}1
    {í}{{\'i}}1
    {ó}{{\'o}}1
    {ú}{{\'u}}1
    {ñ}{{\~n}}1
    {č}{{\v{c}}}1
}
\usepackage[left=3cm,right=3cm,top=3cm,bottom=4cm]{geometry}
\sloppy

\pagestyle{fancy}
\providecommand{\abs}[1]{\lvert#1\rvert}
\providecommand{\norm}[1]{\lVert#1\rVert}

%%% Para las cabeceras
\newcommand{\hsp}{\hspace{20pt}}
\newcommand{\HRule}{\rule{\linewidth}{0.5mm}}
\headheight=50pt
%%% 
\newcommand{\vacio}{\textcolor{white}{ .}}

%%% Para que las ecuaciones se numeren
%%% con el número de sección y el de ecuación.
\renewcommand{\theequation}{\thesection.\arabic{equation}}


% Color azul para algunos 
% textos de la portada
\definecolor{azulportada}{rgb}{0.16, 0.32, 0.75}

%%%% Azul para textos de headings
\definecolor{azulinterior}{rgb}{0.0, 0.2, 0.6}

%%%%%%%%%%%%%%%%%%%%%%%%%%%%%%%%
%%%%%% Datos del proyecto %%%%%%
%%%%%%%%%%%%%%%%%%%%%%%%%%%%%%%%
%%%TÍTULO
%%% Escribirlo en minúsculas, el programa
%%% lo pondrá en mayúsculas en la portada.

\title{Procesamiento de imágenes}

%%%% AUTORES
\author{Lydia Ruiz Martínez \and Pablo Tuñón Laguna}

%%%%%%%%%%%%%%%%%%%%%
%%%%%%%%%%%%%%%%%%%%
\begin{document}

%%%%%%%%%%%%%%%%%%%%%%%%%%%%%%%
%%%%%%%%%%%%%%%%%%%%%%%%%%%%%%%
\begin{titlepage} %%%%% Aquí no hay que tocar nada.
	%%%% Las siguientes instrucciones generarán automáticamente
	%%%% la portada de tu proyecto.
	%%% Cambio de la estructura de esta página
\newgeometry{left=0.6cm,top=1.3cm,bottom=1.2cm}

\fbox{\parbox[c]{18.5cm}{
\begin{center}
\vspace{1.5cm}
{\fontfamily{ptm}\fontsize{24}{28.8}\selectfont{Universidad Pontificia de Comillas}}\\
[3.5em]
{\fontfamily{ptm}\fontsize{24}{5}\selectfont{ICAI}}\\
[4.5em]
{\fontfamily{ptm}\fontsize{28}{5}\selectfont{LABORATORIO 2}}\\
[2cm]
{\fontfamily{ptm}\fontsize{24}{5}\selectfont{Visión por Ordenador I}}\\
[2cm]

% Autor del trabajo de investigación
\textcolor{azulportada}{\fontfamily{ptm}\fontsize{16}{5}\selectfont{\theauthor}}\\
[2cm]
% Título del trabajo
\textcolor{azulportada}
{\fontfamily{ptm}\fontsize{30}{5}\selectfont{\textsc{\thetitle}}}\\
%{\Huge\textbf{\thetitle}}\\
[1.2cm]
\includegraphics[width=10cm]{fonts/Logo ICAI.png}
\\[1.8cm]

{\fontfamily{ptm}\fontsize{16}{5}\selectfont{Curso 2024-2025}}\\
[4cm]
\end{center}
}}
\end{titlepage}
 
 \restoregeometry
 %%%% Volvemos a la estructura de la página normal

%%%%%%%%%%%%%%%%%%%%%%%%%%%%%%

{%\Large

%%%Encabezamiento y pie de página
%%% También se genera automáticamente
%%% Mejor no tocarlo mucho.
\renewcommand{\headrulewidth}{0.5pt}
\fancyhead[R]{
	\textcolor{azulinterior}{\fontfamily{ptm}\fontsize{14}{4}\selectfont{\textbf{\thetitle}}}\\
\textcolor{azulportada}{\fontfamily{ptm}\fontsize{10}{3}\selectfont{Laboratorio 1 de Visión por Ordenador I}}\\
{\fontfamily{ptm}\fontsize{10}{3}\selectfont{\theauthor}}}
\fancyhead[L]{\vacio}

\pagestyle{fancy}
\renewcommand{\footrulewidth}{0.5pt}
\fancyfoot[L]{\footnotesize Universidad Pontificia de Comillas (ICAI) --- curso 2024-2025}
\fancyfoot[C]{\vacio}
\fancyfoot[R]{\footnotesize Página \thepage}

%%%%%%%%%%%%%%%%%%%%
\newpage

\renewcommand{\contentsname}{Índice}
\addtocontents{toc}{\protect\setcounter{tocdepth}{-1}} % Quita el índice de la tabla de contenidos
\tableofcontents
\addtocontents{toc}{\protect\setcounter{tocdepth}{2}}

\newpage

\section{Introducción}

\section{Rabo de gato}



\end{document}



